\documentclass[10pt]{article}

\usepackage{tabularx}
\usepackage{hyperref}
\usepackage{geometry}
\usepackage{titlesec}

\geometry{letterpaper,portrait,margin=0.5in}

%Reduces space between section declarations
\titlespacing*{\section}{0pt}{0ex}{0ex}

%Deletes the page number at the bottom of the page
\pagenumbering{gobble}

\begin{document}
\LARGE\rightline{Stephen Kinser}
\normalsize
\noindent\rightline{1200 North Herndon Street apt 223}
\noindent\rightline{615-972-1663}
\noindent\rightline{sdkinser456@gmail.com}
\noindent\rightline{\href{https://github.com/Riuchando} {https://github.com/Riuchando}}
\hrulefill
\begin{tabularx}{\textwidth}{p{3cm} X}
\section*{Proficiencies} &
% \begin{description}
%     \setlength{\itemsep}{1pt}
%     \item[Scala] Used almost exclusively with the Spark Ecosystem, have also used with grpc
%     \item[Golang] Used in Microservices, streaming jobs, and as an interface with web apis and internal grpc procedures 
%     \item[Python] Used in both personal projects with Keras, scikit learn and qiskit
%     \item[Postgresql] Provide strucuted schemas for the output of ETL jobs and wrote sql queries to satisfy buisness requirements
%     \item[ECS/Kubernetes/Docker] Deployed microservices using both ECS and Kubernetes
%     \item[Protobuf/GRPC] Wrote client and server schemas using protobuf as a framework and grpc as a transport layer
%     \item[Kafka/SQS] for straming applications, have used Kafka and sqs to send data between different services
% \end{description}
\begin{description}
    \setlength{\itemsep}{1pt}
    \item [Languages] Scala, Golang, Python, Java, Rust, Javascript
    \item [Databases] Postgresql, Redis, DynamoDB
    \item [Streaming framework] Kafka, SQS, Kinesis
    \item [Container Orchestration] Kubernetes, ECS
    \item [Service Layer] REST, GRPC 
\end{description}\\
\section*{Experience}

% \section*{Library Software Egineer (Aug 2015 - April 2016)}
% Middle Tennessee State University - MTSU Makers
% \begin{itemize}
% \setlength{\itemsep}{1pt}
% \item Helped establish a Maker Space, by working on both software and hardware projects to engage students outside of class
% \end{itemize}
& \section*{Ironnet Cybersecurity (July 2016 - Current)}
Software Developer
\begin{itemize}
\setlength{\itemsep}{1pt}
\item Optimized Data Science Spark ETL Jobs often improving speed by 10x, and reducing memory usage
\item Translated data science work accross languages to suit company architecture
\item Wrote microservices to interface with Data Science team and web facing ui
% \item Worked on minor networking configuration with nginx and envoy
\end{itemize}
\section*{Research Assistant (Aug 2014 - Aug 2016)}
Middle Tennessee State University - Computer Science Department
\begin{itemize}
\setlength{\itemsep}{1pt}
\item Addressed streamlining and concurrency issues in research of HIV vaccines using Bash and Openmp
\end{itemize}\\
\section*{Projects} &
\section*{Professional}
\begin{description}
    \item [Spark Translation 2016] Translated a Java Spark job to scala spark job, also found points to increase effeciency by 5x
    \item [Unified Health Metrics 2017] using ECS was able to provide a singular place to monitor health of deployed stacks
    \item [LSH Translation 2018] Translated python etl job to Golang to better fit the team's needs
    \item [ML Microservices 2019] Deployed microservices to ecs and provided networking configurations so they could be accessed from internal facing consumers
    \item [Regex Filter Optimization 2019] due to a specific use case was able to remove regex filter and replace it Aho corasick to do substring matching, 
    increasing throughput
\end{description}
\section*{Personal}
\begin{description}
    \setlength{\itemsep}{1pt}
    % \item [Hack Vandy Hackathon 2015] interfaced Google Streetview Javascript API to work with Google Tango motion detection
    % \item [Advent of Code (every year)] every day of december attempt to solve a puzzle using a programming language you have no experience with
    \item [DGA detection 2018] explored different methods of detecting dga (randomly generated domain names) using different types of Neural Networks in keras
    % \item [Discord Bot 2019] worked with a friend on a discord bot in rust
    \item [Qiskit current] decided to learn and contribute to OSS geared to quantum computing
\end{description} \\
\section*{Education} & \section*{Middle Tennessee State University}
B.S. Computer Science (Aug 2011 - April 2016)
\end{tabularx}
\end{document}